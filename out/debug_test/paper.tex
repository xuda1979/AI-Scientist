\documentclass{article}

\usepackage{amsmath, amssymb}
\usepackage{graphicx}
\usepackage{listings}
\usepackage{color}

\lstset{
    language=Python,
    basicstyle=\ttfamily,
    keywordstyle=\color{blue},
    commentstyle=\color{green},
    stringstyle=\color{red},
    numbers=left,
    numberstyle=\tiny
}

\title{Developing a New Algorithm for Test Optimization: A Preliminary Study}
\author{Author Name}
\date{\today}

\begin{document}

\maketitle

\begin{abstract}
This paper presents a preliminary study on the development of a new algorithm for optimizing testing procedures in various fields. While the algorithm is still in its early stages of development, this study aims to provide a theoretical foundation for its design and implementation.
\end{abstract}

\section{Introduction}
The introduction section will provide an overview of the research topic, the motivation for the study, and the objectives of the research.

\section{Related Work}
The related work section will review previous studies on test optimization algorithms, highlighting their strengths and weaknesses, and identifying gaps in the existing literature that this study aims to fill.

\section{Proposed Algorithm}
This section will provide a detailed description of the proposed algorithm, including its design, implementation, and theoretical analysis.

\section{Methodology}
The methodology section will describe the research methods used in the study, including the design of the experiments, the testing procedures, and the evaluation metrics.

\section{Results and Discussion}
The results and discussion section will present the results of the experiments, interpret the results, and discuss their implications. This section will include tables, figures, and graphs to illustrate the results.

\section{Conclusion and Future Work}
The conclusion section will summarize the findings of the study, discuss their implications, and suggest directions for future research.

\bibliographystyle{plain}
\bibliography{references}

\end{document}